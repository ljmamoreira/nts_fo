%\maketitle
\begin{titlepage}
\noindent
{\Huge\sffamily\textsl{Óptica Física}}\\[10mm]
{\Large\textsf{Apontamentos da U.C. Óptica Física
(Optometria e Ciências da Visão)}}\\[5mm]
{\Large\textsf 2017/2018}\\[15mm]
{\large\textsf{Luís J.M. Amoreira (Dep. Física UBI)}}

\vspace{35mm}
\noindent
{\small \textsf{Versão: \today}}
\vfill
\begin{center}
%\includegraphics{figs/cover.pdf}
\end{center}
\vfill
\end{titlepage}
\thispagestyle{plain}
\section*{Introdução}
Estes apontamentos devem ser vistos como um complemento à bibliografia
recomendada. Assim sendo, a sua leitura não substitui o estudo das 
referências sugeridas.

\noindent
Nalguns assuntos, a exposição da matéria é feita aqui de uma maneira que me
parece mais clara. Mas estes apontamentos estão muito incompletos (relativamente
à matéria que lecciono nas aulas) e têm muitos poucos exemplos de aplicação.
Duvido que ultrapassem este estado: hão-de ir melhorando, mas nunca a tempo de
serem verdadeiramente úteis como elemento de estudo único. Assim sendo, caros
alunos, volto a enfatizar: baseiem o vosso estudo nas referências indicadas na
bibliografia, prestem muita atenção nas aulas e tirem delas apontamentos
completos e úteis. Estes apontamentos servirão, quando muito, para dar um apoio
adicional a essas atividades. 
\noindent
Bom estudo!

\vspace{2cm}
\hfill
\begin{minipage}{0.4\linewidth}
UBI, março de 2017\\
LJMA
\end{minipage}

\vspace{3cm}

\section*{Notação}
\begin{itemize}
\item
    Os símbolos matemáticos são representados em itálico como em $m$ ou $\vec
    E$;
\item
    Os símbolos que representam vetores são identificados com uma setinha sobre
    o símbolo, como em $\vec E$ ou $\vec v$
\item
    Os versores (vetores de norma 1) são identificados com um acento circunflexo
    em vez da setinha, como em $\hat r$ ou $\hat n$
\item
    A norma de um vetor representa-se com o mesmo símbolo, mas sem a setinha.
    Por exemplo, a norma do vetor $\vec a$ é $a$. 
\end{itemize}

\vfill
\noindent
{\small
\textsc{Copyleft}\\
Este trabalho pode ser copiado, alterado, vendido, alugado ou oferecido; mas (1)
não pode distribuir trabalhos derivados deste restringindo estes direitos e
(2) deve manter em tais trabalhos derivados uma referência a este trabalho e ao
seu autor. Ou seja, reconhecendo o devido crédito pelo meu trabalho pode fazer
o que quiser com ele, menos impedir terceiros de fazerem o que eles por sua vez
quiserem.\\
\texttt{amoreira@ubi.pt}
}

\pagebreak
\tableofcontents
