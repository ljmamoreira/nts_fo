\chapter{Polarização}
\textsl{{\sffamily(Versão: \today)}}

\noindent
O termo ``polarização'' refere-se à direção da função de onda, quando a onda é
descrita por funções vetoriais. No caso das ondas eletromagnéticas, a
direção do campo elétrico que usamos para descrever a onda (mas o mesmo se passa
com o campo magnético) é perpendicular à direção de propagação, e por isso
dizemos que as ondas eletromagnéticas são transversais. Mesmo estando limitada
ao plano perpendicular à direção de propagação, a polarização de uma onda
eletromagnético pode estar orientada em diferentes direções, e essa orientação
pode variar com o tempo. As possibilidades mais interessantes neste momento são
as da polarização linear e da polarização elíptica, que vamos definir e estudar
já de seguida.

\section{Polarização linear}
\tobedone{}
\section{Polarização elítica}
\tobedone{}
\section{Filtros polarizadores e Lei de Malus}
\tobedone{}
\section{Polarização por reflexão e por difusão}
\tobedone{}
\section{Birefrigência}
\tobedone{}
\section{Aplicações na optometria}
\tobedone{}
